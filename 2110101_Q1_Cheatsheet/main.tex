%! TEX program = xelatex
\documentclass[11pt,a4paper,twocolumn]{article}
\usepackage[a4paper,margin=0.5in]{geometry}
\usepackage{fancyvrb}
% \usepackage[localfonts=true]{tenth}
\usepackage[localfonts=true]{tenth-borworntat}

\renewcommand{\pageheader}{\sffamily\textsb{Python(Q1)} — โดย บวรทัต เด่นดำรงกุล}
\title{Python for Engineering Students (Q1)}
\author{บวรทัต เด่นดำรงกุล \vspace{-5ex}}
\date{}
\setlength{\parindent}{0pt}


\begin{document}

% \maketitle
\thispagestyle{titlepage}
\setstretch{1}

\section*{I/O}
\vspace{-1ex}

print() บรรทัดเดียวกัน
\vspace{-2.5ex}
\begin{verbatim}
print("Hello", end="")
print("World")
\end{verbatim}
\vspace{-1ex}

format string
\vspace{-2.5ex}
\begin{verbatim}
a = "Hello"
b = 32
print("{} {}".format(a, b))
\end{verbatim}
\vspace{-1ex}

รับ Input เป็นจำนวนเต็ม
\vspace{-2.5ex}
\begin{verbatim}
n = int(input())
\end{verbatim}
\vspace{-1ex}

หารเอาแต่เศษ \verb|%|\\
หารไม่เอาเศษ (ปัดทิ้งหมด) \verb|//|\\
ยกกำลัง \verb|**|
\vspace{-1ex}

\vspace{-2ex}
\section*{Math}
\vspace{-1ex}

Import
\vspace{-2.5ex}
\begin{verbatim}
import math
\end{verbatim}
\vspace{-1ex}

ฟังก์ชัน (หน่วยของมุมเป็น radian)
\vspace{-2.5ex}
\begin{verbatim}
print(math.e, math.pi)
print(math.floor(1.2), math.ceil(1.2))
print(math.sin(math.radians(45)))
print(math.log(12, 2))
# log 12 base 2
\end{verbatim}
\vspace{-1ex}

\vspace{-2ex}
\section*{String/List}
\vspace{-1ex}

สร้าง List
\vspace{-2.5ex}
\begin{verbatim}
l = []
\end{verbatim}
\vspace{-1ex}

คำสั่งที่เหมือนกัน
\vspace{-2.5ex}
\begin{verbatim}
print(len(s)) # get length
a = "BB" + "AAA" # a = "BBAAA"
b = [1] * 3 # b = [1, 1, 1]
\end{verbatim}
\vspace{-1ex}

List \& String ตำแหน่งเริ่มที่ \verb|0|
\vspace{-2.5ex}
\begin{verbatim}
a = [5, 2, 1, 3]
print(a[1]) # 2
\end{verbatim}
\vspace{-1ex}

เพิ่มของใน List
\vspace{-2.5ex}
\begin{verbatim}
l = []
l.append(1) # l =[1]
\end{verbatim}
\vspace{-1ex}

ตัด String ด้วยตัวอักษร
\vspace{-2.5ex}
\begin{verbatim}
s = "Hi TT 2"
print(s.split(" ")) # ['Hi', 'TT', '2']
\end{verbatim}
\vspace{-1ex}

Slicing เริ่มที่ \verb|START| จบที่ \verb|STOP| เพิ่มที่ละ \verb|STEP|
\vspace{-2.5ex}
\begin{verbatim}
s = "Grader1"
print(s[1:7:2]) # rdr [START:STOP:STEP]
\end{verbatim}
\vspace{-1ex}

\vspace{-2ex}
\section*{Condition}
\vspace{-1ex}

ตัวอย่างการใช้
\vspace{-2.5ex}
\begin{verbatim}
if x == 1:
    print("x equals 1")
elif x >= 2:
    print("x is more than 1")
else:
    print("x is less than 1")
\end{verbatim}
\vspace{-1ex}

เชื่อมประโยคเงื่อนไข \verb|and| กับ \verb|or|
\vspace{-2.5ex}
\begin{verbatim}
print(12 >= 1 and 12 < 1) # False
print(12 >= 1 or 12 < 1) # True
\end{verbatim}
\vspace{-1ex}

\vspace{-2ex}
\section*{Loop}
\vspace{-1ex}

\verb|while| ใช้ง่ายสุด การใช้เหมือน \verb|if| ทำจนกว่าเงื่อนไขจะผิด
\vspace{-2.5ex}
\begin{verbatim}
i = 1
while i <= 10:
    print(i)
    i += 1
\end{verbatim}
\vspace{-1ex}

\verb|for| ใช้กับ \verb|range| มี \verb|START|, \verb|STOP|, \verb|STEP| เหมือน slicing
\vspace{-2.5ex}
\begin{verbatim}
for i in range(1, 11, 2):
    print(i)
\end{verbatim}
\vspace{-1ex}

\verb|for| แบบใช้กับ List \& String
\vspace{-2.5ex}
\begin{verbatim}
s = "Hello"
for x in s:
    print(x)
\end{verbatim}
\vspace{-1ex}

\vspace{-2ex}
\section*{Trick}
\vspace{-1ex}

Reverse String \& List
\vspace{-2.5ex}
\begin{verbatim}
l = l[::-1]
\end{verbatim}
\vspace{-1ex}

รับ Input บรรทัดละหลายตัว (ได้ List)
\vspace{-2.5ex}
\begin{verbatim}
l = [x for x in input().split()]
\end{verbatim}
\vspace{-1ex}

แปลง Type เอา ชนิดของตัวแปรที่อยากได้ครอบ
\vspace{-2.5ex}
\begin{verbatim}
a = "1234.12"
f = float(a) # convert string to float
\end{verbatim}
\vspace{-1ex}

\begin{center}
\textbf{GRADER 1 TwT}
\end{center}

\end{document}
