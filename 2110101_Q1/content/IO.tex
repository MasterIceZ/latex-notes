\section{Input / Output}

\subsection{Output}

ในการ Output ออกมาทางหน้าจอจะใช้คำสั่ง \verb|print()|

\begin{lstlisting}[language=Python]
print('Hello, World!')
\end{lstlisting}

โดยหากใช้คำสั่ง \verb|print| หลายครั้งจะได้คำในคำสั่ง \verb|print| ออกมาคนละบรรทัดกัน

\begin{lstlisting}[language=Python]
print('Hello')
print('Hello')
\end{lstlisting}

หากต้องการทำให้อยู่บรรทัดเดียวกันใช้คำสั่ง \verb|end=''| มาต่อท้าย

\begin{lstlisting}[language=Python]
print('Hello', end='')
print('World')
\end{lstlisting}

สังเกตว่าหากเปลี่ยนคำใน \verb|''| เป็นคำอื่น ๆ เช่น \verb|'*'| จะได้ \verb|Hello*World| ออกมาแทน

สามารถใช้ \verb|,| ในการคั่นระหว่างสิ่งที่ต้องการแสดงออกมาได้

\begin{lstlisting}[language=Python]
print('Hello', 'World') # Hello World
\end{lstlisting}

จะได้ \verb|Hello World| แต่หากต้องการใช้ตัวอักษรอื่นคั่นระหว่างแต่ละสิ่งที่ต้องการแสดงสามารใช้ \verb|sep='x'| โดย \verb|x| เปลี่ยนได้ตามต้องการ (โดยปกติแล้วเป็น Space)

\begin{lstlisting}[language=Python]
print('Hello', 'World', sep="*") # Hello*World
\end{lstlisting}

\subsection{การ Comment}

ในการ Comment ในภาษา Python ใช้เครื่องหมาย \verb|#| (Sharp, ชาร์ป) ไว้ข้างหน้า

\begin{lstlisting}[language=Python]
print("hello")
# print("hi")
print("world")
\end{lstlisting}

\subsection{ตัวแปร Variable}

ตัวแปร(เบื้องต้น) มีอยู่หลัก ๆ \verb|3| ชนิด คือ

\begin{center}
    \begin{tabular}{|c|c|}
        \hline
        ชนิดของตัวแปร & ชนิดของข้อมูลที่เก็บ\\
        \hline
        \verb|int| & จำนวนเต็ม\\
        \verb|float| & ทศนิยม\\
        \verb|str| & ตัวอักษร (\verb|string|)\\
        \hline
    \end{tabular}
\end{center}

คำสั่ง \verb|type()| สามารถใช้ตรวจสอบชนิดของตัวแปรได้

\begin{lstlisting}[language=Python]
print(type(3))
print(type("Test"))
print(type(321.123))
\end{lstlisting}

ในการสร้างตัวแปรในภาษา Python นั้นสามารถเขียนในรูปแบบ

\begin{verbatim}
<VARIABLE_NAME> = <VALUE>
\end{verbatim}

ตัวอย่างเช่น

\begin{lstlisting}[language=Python]
a = 10
b = "Hello"
print(a, b) # 10 Hello
\end{lstlisting}

ซึ่ง Python เป็นภาษาแบบ Dynamic Type จึงสามารถเปลี่ยนชนิดของตัวแปรได้เรื่อย ๆ

หลักการตั้งชื่อตัวแปรสามารถตั้งชื่อได้ด้วยตัวอักษร ตัวเลข หรือเครื่องหมาย \verb|_| (underscore) โดยตัวเลขไม่สามารถใช้เป็นตัวแรกได้ และชื่อที่ตั้งต้องไม่ซ้ำกับ \href{https://www.w3schools.com/python/python_ref_keywords.asp}{reserved words}

หลักการใช้ \verb|=| คือ \textcolor{red}{เอาค่าทางขวามาใส่ตัวแปรทางซ้าย}

\begin{lstlisting}[language=Python]
a = 3
b = 4
print(a, b) # 3 4
a = b
print(a, b) # 4 4 
\end{lstlisting}

\subsubsection{Type Casting}

ในบางครั้งที่รับประกันได้ว่าสามารถแปลงชนิดของตัวแปรได้สามารถทำการ Type Cast ได้โดยการเอาชนิดที่ต้องการมาครอบ

\begin{lstlisting}[language=Python]
a = "12"
print(type(a)) # <class 'str'>
a = int(a)
print(type(a)) # <class 'int'>
\end{lstlisting}

\subsection{Input}

ในการ Input จะใช้คำสั่ง \verb|input()| โดยจะรับมาทีละบรรทัด

\begin{lstlisting}[language=Python]
s = input()
print("Input:", s)
\end{lstlisting}

\subsection{การดำเนินการบนจำนวนชนิดต่าง ๆ}

สามารถใช้ตัวดำเนินการมาตรฐานได้เช่น \verb|+ - * / ()| และลำดับการทำงานของตัวดำเนินการทั้งหมดจะทำตามหลักคณิตศาสตร์ (คูณ-หาร/บวก-ลบ จากซ้ายไปขวา)

\begin{lstlisting}[language=Python]
a = 10
b = 3
print(a * b) # 30
\end{lstlisting}

\begin{remark}
    \verb|/| จะเป็นการหารแบบได้ทศนิยมออกมาเสมอไม่ว่าจะเป็น \verb|int| หาร \verb|int| ก็ตาม
\end{remark}

การดำเนินการกับตัวแปรเดิมสามารถทำในรูปแบบสั้น ๆ ได้ เช่น

\begin{lstlisting}[language=Python]
a = 10
a += 10 # a = a + 10
a -= 2 # a = a - 2
a *= 4 # a = a * 4
a /= 2 # a = a / 2
\end{lstlisting}

และมีตัวดำเนินการที่ไม่คุ้นเคยเช่น การยกกำลัง การหารเอาเศษ(Modulo) และ การหารแบบไม่เอาเศษเลย(ปัดเศษลงทั้งหมด)

\subsubsection{การยกกำลัง}

การยกกำลังใช้ตัวดำนเนินการ \verb|**|

\begin{lstlisting}[language=Python]
print(10 ** 3) # 1000
\end{lstlisting}

\subsubsection{การหารเอาเศษ Modulo}

การหารเอาเศษจะใช้ตัวดำเนินการ \verb|%|

\begin{lstlisting}[language=Python]
print(10 % 3) # 1
print(10 % 4) # 2
\end{lstlisting}

\subsubsection{การหารแบบไม่เอาเศษและไม่เอาทศนิยม}

การหารเอาเศษจะใช้ตัวดำเนินการ \verb|//|

\begin{lstlisting}[language=Python]
print(10 / 3)  # 3.3333333333333335
print(10 // 3) # 3
\end{lstlisting}

\subsection{Math Module}

ในภาษา Python จะมี Math Module ให้เรียกใช้เพื่อให้สามารถใช้คำสั่งทางคณิตศาสตร์ต่าง ๆ ได้ ก่อนการเรียกใช้จำเป็นจะต้อง \verb|import| เข้ามาก่อนด้วยการ
\begin{lstlisting}[language=Python]
import math
\end{lstlisting}

ตัวอย่างคำสั่งที่น่าสนใจ
\begin{lstlisting}[language=Python]
import math

print(math.sin(0))
print(math.e, math.pi)
print(math.sqrt(9))
print(math.log(4, 2)) # math.log(VALUE, BASE)
\end{lstlisting}

\begin{remark}
    หน่วยของ \verb|math.sin()| และตรีโกณต่าง ๆ เป็น radian หากต้องการแปลงองศาเป็น radian ให้ใช้ \verb|math.radians(degree)| ใส่ใน \verb|math.sin()|
\end{remark}

การหารากที่ \(n\) ของ $x$ สามารถเขียนในรูป
\[
    \sqrt[n]{x} = x^{\frac{1}{n}}
\]
จึงสามารถใช้การยกกำลังเศษส่วนแทนได้
\begin{lstlisting}[language=Python]
print(1000 ** (1 / 3))
\end{lstlisting}