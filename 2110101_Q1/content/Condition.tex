\section{Condition}

\subsection{ประโยคเงื่อนไข}

ประโยคเงื่อนไขคือประโยคที่จะคืนค่า \verb|True| หรือ \verb|False| โดย \verb|True| คือ จริง และ \verb|False| คือ เท็จ

ประโยคเงื่อนไขสามารถเขียนได้โดยง่ายด้วย \verb|> < >= <=| แต่หากต้องการตรวจสอบว่าเท่ากับหรือไม่ต้องใช้ \verb|==| ไม่สามารถใช้ \verb|=| ตัวเดียวได้ เนื่องจาก \verb|=| เป็นการเอาค่าทางขวามาใส่ทางซ้าย

ในการตรวจสอบว่าตัวแปรสองตัวมีค่าไม่เท่ากันหรือไม่ใช้ \verb|!=| ในการตรวจสอบ

\begin{lstlisting}[language=Python]
print(1 < 2) # True
print(2 >= 3) # False
a = 2
b = 3
print(a == b) # False
print(a != b) # True
print(a == 2) # True
\end{lstlisting}

ในใส่นิเสธ การกลับประโยคจากจริงเป็นเท็จ เท็จเป็นจริง หรือการ negation สามารถใช้คำสั่ง \verb|not|

\begin{lstlisting}[language=Python]
a = 1
print(not (a == 1)) # False
print(not (a > 1)) # True
\end{lstlisting}

ประโยคเงื่อนไขสามารถเขียนเป็นช่วงก็ได้

\begin{lstlisting}[language=Python]
a = 2
print(1 <= a <= 3) # True
print(-1 <= a <= 1) # False
\end{lstlisting}

ในการตรวจสอบว่าค่านั้น ๆ อยู่ใน List หรือไม่ใช้คำสั่ง \verb|in|

\begin{lstlisting}[language=Python]
l = [1, 2, 3, 4, 5]
print(2 in l) # True
print(7 in l) # False
\end{lstlisting}

การเปรียบเทียบ String จะเปรียบเทียบด้วยลำดับดังนี้

\begin{enumerate}
    \item ตัวอักษรตัวพิมพ์ใหญมีค่าน้อยกว่าตัวพิมพ์เล็ก
    \item ตัวอักษรมีค่ามากน้อยเรียงตามภาษาอังกฤษ
    \item หากเป็นตัวเลขมีค่าตามตัวนั้น ๆ
    \item จะเปรียบเทียบทีละตัวจากซ้ายไปขวา
    \item หากเปรียบเทียนไปเรื่อย ๆ จนมีฝั่งใดฝั่งหนึ่งไม่เหลือตัวอักษรให้พิจารณาแล้วจะให้ตัวที่หมดก่อนน้อยกว่า
\end{enumerate}

\begin{lstlisting}[language=Python]
print("A" < "C") # True
print("ABC" < "aA") # True
print("bbbb" < "aaaa") # False
\end{lstlisting}

\subsection{การเชื่อมประโยคเงื่อนไข}

ในการเชื่อมประโยคเงื่อนไขจะมีการ และ(AND) และ หรือ(OR) โดยหลักการเหมือนตรรกศาสตร์ คือ หากเป็นการ AND จะต้องจริงทั้งคู่ถึงจะจริง และ OR เป็นจริงเพียงฝั่งใดฝั่งหนึ่งก็พอ

หลักการเขียน AND และ OR

\begin{verbatim}
CONDITION1 and CONDITION2
CONDITION1 or CONDITION2
\end{verbatim}

\subsection{If}

รูปแบบการเขียน If สามารถเขียนได้โดย

\begin{verbatim}
if CONDITION:
    DO SOMETHING
\end{verbatim}

เมื่อ \verb|CONDITION| คือ เงื่อนไขที่ต้องการ และสังเกตว่าคำสั่งที่ต้องการทำเมื่อเงื่อนไขถูกจะถูก TAB มา 1 ครั้ง

ประโยคเงื่อนไขแบบ If จะทำงานเมื่อเงื่อนไขที่ตั้งไว้ถูก โดยคำสั่งที่ต้องการทำเมื่อใช้คำสั่ง If 

\begin{lstlisting}[language=Python]
A = 3
B = 2
if A > B:
    print("A > B")
\end{lstlisting}

แต่หากไม่ถูกจะไม่ทำอะไร

\subsection{Else}

รูปแบบการเขียน Else สามารถเขียนได้โดย 

\begin{verbatim}
if CONDITION:
    DO SOMETHING
else:
    DO SOMETHING
\end{verbatim}

สังเกตว่าต้องมี If ขึ้นก่อนเสมอและสำหรับ 1 If มีได้เพียง 1 Else เท่านั้น

หลักการทำงานของ Else คือหากเงื่อนไขใน If ไม่ถูกสามารใช้ Else มาเพื่อรองรับในกรณีที่ไม่ถูกได้

\begin{lstlisting}[language=Python]
A = 2
B = 3
if A > B:
    print("A > B")
else:
    print("A < B")
\end{lstlisting}

\subsection{Elif}

รูปแบบการเขียน Elif (else if) สามารถเขียนได้โดย 

\begin{verbatim}
if CONDITION:
    DO SOMETHING
elif CONDITION2:
    DO SOMETHING
else:
    DO SOMETHING
\end{verbatim}

ในการใช้ Elif สามารถใช้ได้หลังมี If เท่านั้นและต้องอยู่ก่อน Else (ถ้ามี)

Else if สามารถมีกี่ตัวก็ได้เหมือนเงื่อนไขรอง ๆ ลงมาจนกว่าจะถึง Else

\begin{lstlisting}[language=Python]
A = 3
B = 3
if A > B:
    print("A > B")
elif A == B:
    print("A is equals to B")
else:
    print("A < B")
\end{lstlisting}
