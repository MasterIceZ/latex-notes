\section{String \& List}

\subsection{String}

String คือสายอักขระ (ตัวแปรประเภท \verb|str|) เช่น "123", "Hello"

\subsubsection{String Operations}

การหาความยาวใช้่คำสั่ง \verb|len()|

\begin{lstlisting}[language=Python]
s = "Hi"
print(len(s)) # 2
t = "Hello"
print(len(t)) # 5
\end{lstlisting}

การต่อ String 2 Strings เข้าด้วยกันใช้การบวกได้เลย

\begin{lstlisting}[language=Python]
s = "123" + "321"
print(s) # 123321
\end{lstlisting}

ในการต่อกันซ้ำ ๆ สามารถใช้คำสั่ง \verb | * n| เมื่อ \verb|n| เป็นจำนวนครั้งได้

\begin{lstlisting}[language=Python]
a = "51" * 10
print(a) # 51515151515151515151
\end{lstlisting}

\subsubsection{String Indexing}

String เก็บตัวอักษรหลายตัวก็จริง แต่จริง ๆ แล้ว String สร้างกล่องขึ้นมาจำนวนเท่ากับความยาวเพื่อให้่แต่ละกล่องเก็บตัวอักษร 1 ตัวเท่านั้น

เราจึงสามารถใช้ประโยชน์จากสิ่งนี้ในการเรียกแค่ตัวอักษรบางตัวของ String ได้ โดยรูปแบบการเรียกจะใช้

\begin{center}
    \begin{tabular}{|c|c|c|c|c|c|c|}
         \hline
         String & P & y & t & h & o & n\\
         \hline
         Index (Forward) & 0 & 1 & 2 & 3 & 4 & 5\\
         \hline
         Index (Backward) & -6 & -5 & -4 & -3 & -2 & -1\\
         \hline
    \end{tabular}
\end{center}

สังเกตว่า Index แบบ Forward จะมีค่าตั้งแต่ \verb|0| ถึง \verb|len(s) - 1| เมื่อ \verb|s| คือสตริงและ Backward มีค่าตั้งแต่ \verb|-len(s)| ถึง \verb|-1|

ในการเรียก Index จะใช้ \verb|[i]| เมื่อ \verb|i| คือ Index ที่ต้องการเรียก

\begin{lstlisting}[language=Python]
a = "Python"
print(a[0], a[-4]) # P t
\end{lstlisting}

\subsubsection{String Slicing}

การ Slicing คือการ ``หั่น" String ออกมาด้วยวิธีการคล้าย ๆ วน Loop โดยมีรูปแบบการเขียนคือ \verb|[start:stop:step]| โดยการเลือกของ String จะเลือกจาก Index \verb|start| ทุก ๆ \verb|step| ตัวไปจนกว่าจะเท่ากับหรือมากกว่า \verb|stop|

\begin{center}
    \begin{tabular}{|c|c|c|c|c|c|c|}
         \hline
         String & P & y & t & h & o & n\\
         \hline
         Index (Forward) & 0 & 1 & 2 & 3 & 4 & 5\\
         \hline
         Index (Backward) & -6 & -5 & -4 & -3 & -2 & -1\\
         \hline
    \end{tabular}
\end{center}

โดยหากเว้น \verb|start| ไว้จะถือว่าเป็น \verb|0|, เว้น \verb|stop| ไว้จะถือว่าเป็นขนาดของ String, เว้น \verb|step| ไว้จะถือว่าเป็น \verb|1|

\begin{lstlisting}[language=Python]
a = "Python"
print(a[0:3:1]) # Pyt
print(a[1::2]) # yhn
print(a[::3]) # Ph
\end{lstlisting}

แต่ถ้าค่า \verb|step| ติดลบ หากเว้น \verb|start| ไว้จะถือว่าเป็น \verb|-1|, เว้น \verb|stop| ไว้จะถือว่าเป็นขนาดของ \verb|-len(s)-1| เมื่อ \verb|s| คือ String ที่ต้องการ Slice

\begin{lstlisting}[language=Python]
a = "Python"
print(a[::-1]) # nohtyP
print(a[-1:-4:-2]) # nh
\end{lstlisting}

\subsection{List}

List คือกล่องเก็บข้อมูลที่สามารถเก็บข้อมูลหลาย ๆ ชนิดไว้ได้

\subsubsection{การสร้าง List}

การสร้่าง List เปล่าไม่มีอะไรเลย สามารถทำได้โดย

\begin{lstlisting}[language=Python]
l = list()
# or
l = []
\end{lstlisting}

การสร้่าง List แบบมีบางอย่างอยู่ข้างในตั้งแต่แรก สามารถทำได้โดย

\begin{lstlisting}[language=Python]
l = [3, "hi", 12.3]
\end{lstlisting}

\subsubsection{List Operations}

\textcolor{cyan}{คำสั่งส่วนมากคล้าย String}

การหาความยาวใช้่คำสั่ง \verb|len()|

\begin{lstlisting}[language=Python]
l = [3, "hi", 12.3]
print(len(l)) # 3
\end{lstlisting}

การต่อ List 2 Lists เข้าด้วยกันใช้การบวกได้เลย

\begin{lstlisting}[language=Python]
s = [3, "hi", 12.3] + [1, "t"]
print(s) # [3, "hi", 12.3, 1, "t"]
\end{lstlisting}

ในการต่อกันซ้ำ ๆ สามารถใช้คำสั่ง \verb | * n| เมื่อ \verb|n| เป็นจำนวนครั้งได้

\begin{lstlisting}[language=Python]
a = [3] * 5
print(a) # [3, 3, 3, 3, 3]
\end{lstlisting}

\subsubsection{List Indexing}

List จะเก็บข้อมูลคล้าย ๆ กับ String โดยจะเก็บเป็นกล่อง ๆ 

\begin{center}
    \begin{tabular}{|c|c|c|c|c|c|c|}
         \hline
         List & 31 & ``Hello" & 123.3 & ``Test" & 3.14 & 999\\
         \hline
         Index (Forward) & 0 & 1 & 2 & 3 & 4 & 5\\
         \hline
         Index (Backward) & -6 & -5 & -4 & -3 & -2 & -1\\
         \hline
    \end{tabular}
\end{center}

หลักการใช้เหมือนกับ String

\begin{lstlisting}[language=Python]
l = [31, "Hello", 123.3, "Test", 3.14, 99]
print(a[2]) # Hello
\end{lstlisting}

\subsubsection{List Slicing}

\textit{เหมือนกับ String}

\subsubsection{การเปลี่ยนค่าข้างใน List}

สามารถใช้หลักการของ \verb|=| ได้เลย

\begin{lstlisting}[language=Python]
l = [31, "Hello", 123.3, "Test", 3.14, 99]
l[1] = -1
print(l) # [31, -1, 123.3, "Test", 3.14, 99]
\end{lstlisting}

\begin{remark}
    ตัวแปรประเภท String ไม่สามารถแก้ไขค่าในช่องใดช่องหนี่งเหมือน List ได้
\end{remark}

\subsubsection{การใส่ค่าเพิ่มใน List}

การใส่ค่าเข้าไปเพิ่มใน List จะใช้ method \verb|append()| เพื่อใส่ค่าเข้าไป

\begin{lstlisting}[language=Python]
l = [1, "T"]
l.append('12')
print(l) # [1, 'T', '12']
\end{lstlisting}

\subsection{การแยก String ออกเป็น List}

ใช้่ method \verb|split()| ของ String เพื่อทำการตัด String ออกมาเป็น List โดยจะตัดด้วย Space โดยจะได้ List of Strings ออกมา

\begin{lstlisting}[language=Python]
s = "Hello World Hi 1234"
l = s.split()
print(l) # ["Hello", "World", "Hi", "1234"]
\end{lstlisting}

แต่หากต้องการตัดด้วยตัวอักษรอื่น ๆ ให้ใส่ตัวนั้น ๆ ในวงเล็บ

\begin{lstlisting}[language=Python]
s = "Hello*World*Hi"
l = s.split("*")
print(l) # ["Hello", "World", "Hi"]
\end{lstlisting}